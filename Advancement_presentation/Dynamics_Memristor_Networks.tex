\documentclass[mathserif]{beamer}
\usepackage{amsmath}
\mode<presentation>
\usetheme{Madrid}
\AtBeginSection[]{}
\title{Phase Transitions and Scale Free Phenomena in Networks of Memristive Elements}
\author{Forrest Sheldon}
\institute{UCSD}
\date{August 7, 2015}
\titlegraphic{\includegraphics[width=2.5cm]{gl-1-logo.png}}
\begin{document}

\begin{frame}
\titlepage
\end{frame}

\begin{frame}
\frametitle{A Perspective On Memristors}

\begin{columns}
\column{0.5\textwidth}
\begin{itemize}
\item Simple(est) Nanomachines
\item Sit between scales
\item Wide variety of behaviors: ON/OFF ratio, volatility, timescales, programmability

\end{itemize}
\column{0.5\textwidth}
\includegraphics[width=5cm]{Granular_Material.png}
\end{columns}
\end{frame}

\begin{frame}
\frametitle{Example: $Ag|Ag_2 S|Ag$ Atomic Switch}
\begin{center}
\includegraphics[width=10cm]{Atomic_Switch_Filament.png}
\end{center}
\begin{itemize}
\item Rely on Ionic motion
\item Strong threshold effect from $Ag_2 S$ phase transition
\item Device Size $\approx 100$nm, Switching Times $~10^{-7}$s or less
\item Competition between thermal effects and dynamics allow them to show short term and long term potentiation-like behaviors,
operating in stochastic and non-volatile regimes, and can store multiple states in quantized conductance, sharp on/off transition
\end{itemize}
\end{frame}

\begin{frame}
\frametitle{Atomic Switch Networks}

\begin{center}
\includegraphics[width=9cm]{ASN_fabrication.png}
\end{center}
\begin{itemize}
\item Show robust memristive switching with much larger size and voltages than
nanoscale memristors
\item Experiments and theoretical studies have pointed towards possible first order phase
transition in their cycling and possible criticality in fluctuations.
\end{itemize}

\end{frame}

\begin{frame}
\frametitle{Questions}
\begin{itemize}
\item How can we see the memristive cycling, which is a collective action of the network,
in the language of phase transitions?
\item How do the parameters of the devices affect the presence of a transition?
\item The full time integration is too difficult to discern a transition or avalanches.
What features can we discard?
\item Can these networks display SOC?
\end{itemize}
\end{frame}

\begin{frame}
\frametitle{Memristor Network Model}
\begin{itemize}
\item Memristive switching is fast compared to external control
but slow compared to electrons $\to$ at any time point, networks are
ohmic but individual memristors switch discontinuously between
$g_{OFF}$ and $g_{ON}$ (Possible Caveats)
\item Thresholds are essential to causal structure of switching
\item Most behaviors are robust to network structure $\to$ pick any canonical disordered network
\item Disorder induces a voltage distribution over the network $\to$ absorb voltage disorder into
thresholds and simulate on regular lattice with disordered thresholds
\item Memristors act like switches that transition from a thermodynamically favored OFF state
to ON when the current passes some random threshold $t_i$ drawn from a distribution $p(t)$
\end{itemize}

\end{frame}

\begin{frame}
\frametitle{Simulation Results}
\begin{itemize}
\item $g_{ON} / g_{OFF} = 2$
\includegraphics[width=0.8\textwidth]{ON2_run.png}
\item $g_{ON} / g_{OFF} = 100$
\includegraphics[width=0.8\textwidth]{ON100_run.png}
\item $g_{ON} / g_{OFF} = 1000$
\includegraphics[width=0.8\textwidth]{ON1000_run.png}
\end{itemize}
\end{frame}

\begin{frame}
\frametitle{Internal Dynamics}
\begin{center}
\includegraphics[width=0.8\textwidth]{Inside_a_network.png}
\end{center}
\end{frame}

\begin{frame}
\frametitle{Mean-Field Theory}
\begin{itemize}
\item Begin with $$P = \sum_i \sigma_i \Delta V_i^2$$
\item Require that the total power dissipated in the mean field model be the same
as that dissipated in the network,
$$P = G(\{\sigma_i\})V^2$$
\item Use effective medium theory to write $G(\{\sigma_i\})\approx G(\phi)$ for $\phi = \frac{\sum_i \sigma_i}{N}$
\item Insert a $\phi$ to get it back to it's original form,
$$P = \sum_i \sigma_i \frac{G(\phi)V^2}{\phi N}$$
\end{itemize}

\end{frame}

\begin{frame}
\frametitle{Mean-Field Theory}
\begin{itemize}
\item Our mean field voltage is thus
$\Delta V_{MF} = \sqrt{\frac{G(f)}{\phi(f)}}\frac{V}{\sqrt{N}}$
where I've traded $\phi$ for $f = \frac{n_{ON}}{N}$
\item From this we form a self-consistency requirement: the fraction of memristors
that have switched to $G_{ON}$ must be the fraction whose threshold is below the MF
current $g_{OFF} \sqrt{\frac{G(f)}{\phi(f)}}\frac{V}{\sqrt{N}} = h(f)v$,
$$f = \int_0^{h(f)v} p(t) dt$$
\end{itemize}
\begin{center}
\includegraphics[width=0.6\textwidth]{MF_voltage.png}
\end{center}
\end{frame}

\begin{frame}
\frametitle{One-Dimensional Networks}
\begin{itemize}
\item In 1D, we can do this exactly!
\item For a 1D chain of conductors in series, $G(f) = \frac{g_{ON}g_{OFF}}{N(g_{ON} - f (g_{ON} - g_{OFF}))}$
$$f = \int_0^{G(f)V} p(t) dt$$
\end{itemize}
\begin{columns}
\column{0.5\textwidth}
\centering
\includegraphics[width=\textwidth]{1D_Network_Cond.png}

\column{0.5\textwidth}
\centering
\includegraphics[width=0.9\textwidth]{SC_1D_ON100_expon.png}

\end{columns}
\end{frame}

\begin{frame}
\frametitle{Finding the transition}
First order phase transition at the solution of:
$$f = \int_0^{G(f)V} p(t) dt,  \quad 1 = p(G(f)V)G'(f)V \quad 0 \le f \le 1$$
For Uniform(0,1) this gives, 
$f_b = \frac{1}{2(1-\alpha)}, \: V_b = \frac{TN}{4g_{OFF}(1-\alpha)},
 \: \alpha = \frac{g_{OFF}}{g_{ON}}$
\centering
\includegraphics[width=\textwidth]{SC_1D_Uniform01.png}
\end{frame}

\begin{frame}
\frametitle{Finding the Transition}
\centering
\includegraphics[width=0.8\textwidth]{1D_Cond.png}

\end{frame}

\begin{frame}
\begin{itemize}
\item Phase Transitions in Current Controlled Networks
\item Oscillations in Avalanches
\end{itemize}
\end{frame}

\end{document}
