%% ****** Start of file apstemplate.tex ****** %
%%
%%
%%   This file is part of the APS files in the REVTeX 4 distribution.
%%   Version 4.1r of REVTeX, August 2010
%%
%%
%%   Copyright (c) 2001, 2009, 2010 The American Physical Society.
%%
%%   See the REVTeX 4 README file for restrictions and more information.
%%
%
% This is a template for producing manuscripts for use with REVTEX 4.0
% Copy this file to another name and then work on that file.
% That way, you always have this original template file to use.
%
% Group addresses by affiliation; use superscriptaddress for long
% author lists, or if there are many overlapping affiliations.
% For Phys. Rev. appearance, change preprint to twocolumn.
% Choose pra, prb, prc, prd, pre, prl, prstab, prstper, or rmp for journal
%  Add 'draft' option to mark overfull boxes with black boxes
%  Add 'showpacs' option to make PACS codes appear
%  Add 'showkeys' option to make keywords appear
\documentclass[aps,prl,preprint,groupedaddress]{revtex4-1}
%\documentclass[aps,prl,preprint,superscriptaddress]{revtex4-1}
%\documentclass[aps,prl,reprint,groupedaddress]{revtex4-1}

% You should use BibTeX and apsrev.bst for references
% Choosing a journal automatically selects the correct APS
% BibTeX style file (bst file), so only uncomment the line
% below if necessary.
%\bibliographystyle{apsrev4-1}

\begin{document}

% Use the \preprint command to place your local institutional report
% number in the upper righthand corner of the title page in preprint mode.
% Multiple \preprint commands are allowed.
% Use the 'preprintnumbers' class option to override journal defaults
% to display numbers if necessary
%\preprint{}

%Title of paper
\title{Phase Transitions in Networks of Memristive Elements}

% repeat the \author .. \affiliation  etc. as needed
% \email, \thanks, \homepage, \altaffiliation all apply to the current
% author. Explanatory text should go in the []'s, actual e-mail
% address or url should go in the {}'s for \email and \homepage.
% Please use the appropriate macro foreach each type of information

% \affiliation command applies to all authors since the last
% \affiliation command. The \affiliation command should follow the
% other information
% \affiliation can be followed by \email, \homepage, \thanks as well.
\author{Forrest Sheldon}
\email[]{fsheldon@physics.ucsd.edu}
%\homepage[]{Your web page}
%\thanks{}
%\altaffiliation{}
\author{Massimiliano Di Ventra}
\email[]{diventra@physics.ucsd.edu}
\affiliation{UCSD}

%Collaboration name if desired (requires use of superscriptaddress
%option in \documentclass). \noaffiliation is required (may also be
%used with the \author command).
%\collaboration can be followed by \email, \homepage, \thanks as well.
%\collaboration{}
%\noaffiliation

\date{\today}

\begin{abstract}
Memristive elements display many behaviors analogous to those found in
organic synapses such as short-term and long-term potentiation. As such,
the development of neuromorphic devices utilizing these behaviors is an
active area of research. Of particular
interest to this investigation are devices that attempt to combine the
dynamics of synapses with the complex structure of biological neural
tissue. Experimental systems accomplishing have begun to emerge but
little is known
about the dynamics of memristive elements in networks and in the presence
structural disorder. We propose a model for disordered memristive networks 
in
the limit of a slowly ramped voltage and show through simulations that these
produce a first-order phase transition in the conductivity for sufficiently high
values of the ON/OFF ratio.  A mean-field theory that duplicates many features of
the transition is derived and examined with particular attention paid to the role
of boundary conditions and the potential for bipolar memristive switching.  The
dynamics of the mean-field theory suggest a distribution of conductance jumps which
may be accessible experimentally and finally the relevance to device design and
their ability to support computation is discussed.

\end{abstract}

% insert suggested PACS numbers in braces on next line
\pacs{}
% insert suggested keywords - APS authors don't need to do this
%\keywords{}

%\maketitle must follow title, authors, abstract, \pacs, and \keywords
\maketitle


\end{document}
%
% ****** End of file apstemplate.tex ******

