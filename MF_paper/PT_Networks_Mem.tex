%% ****** Start of file apstemplate.tex ****** %
%%
%%
%%   This file is part of the APS files in the REVTeX 4 distribution.
%%   Version 4.1r of REVTeX, August 2010
%%
%%
%%   Copyright (c) 2001, 2009, 2010 The American Physical Society.
%%
%%   See the REVTeX 4 README file for restrictions and more information.
%%
%
% This is a template for producing manuscripts for use with REVTEX 4.0
% Copy this file to another name and then work on that file.
% That way, you always have this original template file to use.
%
% Group addresses by affiliation; use superscriptaddress for long
% author lists, or if there are many overlapping affiliations.
% For Phys. Rev. appearance, change preprint to twocolumn.
% Choose pra, prb, prc, prd, pre, prl, prstab, prstper, or rmp for journal
%  Add 'draft' option to mark overfull boxes with black boxes
%  Add 'showpacs' option to make PACS codes appear
%  Add 'showkeys' option to make keywords appear
\documentclass[aps,prl,preprint,groupedaddress]{revtex4-1}
%\documentclass[aps,prl,preprint,superscriptaddress]{revtex4-1}
%\documentclass[aps,prl,reprint,groupedaddress]{revtex4-1}

% You should use BibTeX and apsrev.bst for references
% Choosing a journal automatically selects the correct APS
% BibTeX style file (bst file), so only uncomment the line
% below if necessary.
%\bibliographystyle{apsrev4-1}

\begin{document}

% Use the \preprint command to place your local institutional report
% number in the upper righthand corner of the title page in preprint mode.
% Multiple \preprint commands are allowed.
% Use the 'preprintnumbers' class option to override journal defaults
% to display numbers if necessary
%\preprint{}

%Title of paper
\title{Phase Transitions in Networks of Memristive Elements}

% repeat the \author .. \affiliation  etc. as needed
% \email, \thanks, \homepage, \altaffiliation all apply to the current
% author. Explanatory text should go in the []'s, actual e-mail
% address or url should go in the {}'s for \email and \homepage.
% Please use the appropriate macro foreach each type of information

% \affiliation command applies to all authors since the last
% \affiliation command. The \affiliation command should follow the
% other information
% \affiliation can be followed by \email, \homepage, \thanks as well.
\author{Forrest C. Sheldon}
\email[]{fsheldon@ucsd.edu}
%\homepage[]{Your web page}
%\thanks{}
%\altaffiliation{}
\affiliation{UCSD}

\author{Massimiliano Di Ventra}
\email[]{diventra@ucsd.edu}
%\homepage[]{Your web page}
%\thanks{}
%\altaffiliation{}
\affiliation{UCSD}

%Collaboration name if desired (requires use of superscriptaddress
%option in \documentclass). \noaffiliation is required (may also be
%used with the \author command).
%\collaboration can be followed by \email, \homepage, \thanks as well.
%\collaboration{}
%\noaffiliation

\date{\today}

\begin{abstract}
% insert abstract here
\end{abstract}

% insert suggested PACS numbers in braces on next line
\pacs{}
% insert suggested keywords - APS authors don't need to do this
%\keywords{}

%\maketitle must follow title, authors, abstract, \pacs, and \keywords
\maketitle

% body of paper here - Use proper section commands
% References should be done using the \cite, \ref, and \label commands
%\section{}
% Put \label in argument of \section for cross-referencing
%\section{\label{}}
%\subsection{}
%\subsubsection{}

\section{INTRODUCTION}

Memristors are two terminal passive devices that display pinched hysteresis
in their I-V curve.  

The impending limits of our current computational paradigm have galvanized
interest in alternative forms of computation. Of particular interest to the
authors are attempts to address the von Neumann Bottleneck \cite{Backus1978}
whereby increases in processor speed are muted by the necessity of
shuffling data and addresses back and forth between the processor and memory.
This necessity of the von Neumann architecture suggests a solution in which
computation is performed directly in memory.  Physically coupled memory units
that dynamically were driven into the solution of a computational problem
would conceivably circumvent this problem (or at least translate it into the
problem of transmitting information across it's extent) but the realization
of such an architecture is a foreboding problem.

In this however, we are buoyed by the fact that the nervous system seems to
compute in just this manner; we perceive no separate logical and memory
faculties that information is shuttled between in the brain.  Memory seems
to be stored
in the strength of synaptic coupling between neurons and by the neuronal
dynamics, the synaptic couplings are modified and computation is performed.
The desire to mimic these properties in hardware is an active field of
research in which memristors have begun to play a significant role.  
memristor based associative memories, neural networks, cellular automata,
and many other paradigms are being explored as ways of realizing 

The impending limits of our current computational paradigm have galvanized
interest in alternative forms of computation. As transistor densities have
increased, limits on transistor size and increased energy consumption have
spurred interest in electronic components that function reliably at small
scales and store their states passively.  Similarly, limits due to our
computational model such as the von Neumann Bottleneck have generated
interest in alternative schemes that do computation directly in memory.
CITE

Memristor based solutions have been proposed to meet both of these needs.
They are
necessarily nanoscale devices that passively maintain their resistance
state and can function both as logical and memory components.  As such,
several emerging technologies have centered around the memristor as
their core computational unit.  Among these are the crossbar array,
memristor neural networks and associative memories, memristive cellular
automata, and memcomputing.  While the conceptual and engineering
challenges to realizing new computational paradigms are great, we are
buoyed by the fact that the nervous system appears to perform computation
in just this way, with no discernable separation between memory and
logical faculties.  It would seem that memories are stored in a
distributed manner, within the synaptic connections of many neurons
and computation occurs through the generation of action potentials that
propogate through the system, modifying synaptic coupling globally.
Memristors' ability to mimic the learning properties of synapses such as
short-term and long-term memory, have further propelled their standing as
a possible hardware element for neuromorphic computing.

While most neuromorphic computing solutions have advocated a bottom-up
approach, designing fundamental units such as CMOS neurons
and linking them together in a
particular configuration to solve problems, a few have advocated a top
down approach.  In such an approach, the gross features of nervous
tissue are duplicated: a disordered, recurrent network with memory, and
the resulting system is studied to see what behaviors emerge.  As disorder
and noise play a strong role in the nervous system, we suspect a statistical
character to the dynamics and as such we investigate the statistical
dynamics of 

In order to highlight the general features of such networks, we take as
a concrete example the Atomic Switch Networks of Stieg et al. CITE.

\section{ATOMIC SWITCH NETWORKS}

In order to achieve a complex disordered network architecture, the
authors use an electroless deposition process in which solution phase
$Ag^+$ exchanges with solid $Cu$.  By seeding an area with $Cu$ 
microspheres or seed posts, the depositing $Ag$ is led to form a
densely interconnected network of wires

\section{MODEL \& METHODS}


\section{SIMULATIONS}


\section{MEAN-FIELD THEORY}


\section{DISCUSSION}


% If in two-column mode, this environment will change to single-column
% format so that long equations can be displayed. Use
% sparingly.
%\begin{widetext}
% put long equation here
%\end{widetext}

% figures should be put into the text as floats.
% Use the graphics or graphicx packages (distributed with LaTeX2e)
% and the \includegraphics macro defined in those packages.
% See the LaTeX Graphics Companion by Michel Goosens, Sebastian Rahtz,
% and Frank Mittelbach for instance.
%
% Here is an example of the general form of a figure:
% Fill in the caption in the braces of the \caption{} command. Put the label
% that you will use with \ref{} command in the braces of the \label{} command.
% Use the figure* environment if the figure should span across the
% entire page. There is no need to do explicit centering.

% \begin{figure}
% \includegraphics{}%
% \caption{\label{}}
% \end{figure}

% Surround figure environment with turnpage environment for landscape
% figure
% \begin{turnpage}
% \begin{figure}
% \includegraphics{}%
% \caption{\label{}}
% \end{figure}
% \end{turnpage}

% tables should appear as floats within the text
%
% Here is an example of the general form of a table:
% Fill in the caption in the braces of the \caption{} command. Put the label
% that you will use with \ref{} command in the braces of the \label{} command.
% Insert the column specifiers (l, r, c, d, etc.) in the empty braces of the
% \begin{tabular}{} command.
% The ruledtabular enviroment adds doubled rules to table and sets a
% reasonable default table settings.
% Use the table* environment to get a full-width table in two-column
% Add \usepackage{longtable} and the longtable (or longtable*}
% environment for nicely formatted long tables. Or use the the [H]
% placement option to break a long table (with less control than 
% in longtable).
% \begin{table}%[H] add [H] placement to break table across pages
% \caption{\label{}}
% \begin{ruledtabular}
% \begin{tabular}{}
% Lines of table here ending with \\
% \end{tabular}
% \end{ruledtabular}
% \end{table}

% Surround table environment with turnpage environment for landscape
% table
% \begin{turnpage}
% \begin{table}
% \caption{\label{}}
% \begin{ruledtabular}
% \begin{tabular}{}
% \end{tabular}
% \end{ruledtabular}
% \end{table}
% \end{turnpage}

% Specify following sections are appendices. Use \appendix* if there
% only one appendix.
%\appendix
%\section{}

% If you have acknowledgments, this puts in the proper section head.
%\begin{acknowledgments}
% put your acknowledgments here.
%\end{acknowledgments}

% Create the reference section using BibTeX:
\bibliography{/home/fsheldon/Documents/BibTeX/Advancement_PT_Memnets}

\end{document}
%
% ****** End of file apstemplate.tex ******

