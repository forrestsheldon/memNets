%% ****** Start of file apstemplate.tex ****** %
%%
%%
%%   This file is part of the APS files in the REVTeX 4 distribution.
%%   Version 4.1r of REVTeX, August 2010
%%
%%
%%   Copyright (c) 2001, 2009, 2010 The American Physical Society.
%%
%%   See the REVTeX 4 README file for restrictions and more information.
%%
%
% This is a template for producing manuscripts for use with REVTEX 4.0
% Copy this file to another name and then work on that file.
% That way, you always have this original template file to use.
%
% Group addresses by affiliation; use superscriptaddress for long
% author lists, or if there are many overlapping affiliations.
% For Phys. Rev. appearance, change preprint to twocolumn.
% Choose pra, prb, prc, prd, pre, prl, prstab, prstper, or rmp for journal
%  Add 'draft' option to mark overfull boxes with black boxes
%  Add 'showpacs' option to make PACS codes appear
%  Add 'showkeys' option to make keywords appear
%\documentclass[aps,prl,preprint,groupedaddress]{revtex4-1}
%\documentclass[aps,prl,preprint,superscriptaddress]{revtex4-1}
\documentclass[aps,prl,reprint,groupedaddress]{revtex4-1}

% You should use BibTeX and apsrev.bst for references
% Choosing a journal automatically selects the correct APS
% BibTeX style file (bst file), so only uncomment the line
% below if necessary.
%\bibliographystyle{apsrev4-1}

\begin{document}

% Use the \preprint command to place your local institutional report
% number in the upper righthand corner of the title page in preprint mode.
% Multiple \preprint commands are allowed.
% Use the 'preprintnumbers' class option to override journal defaults
% to display numbers if necessary
%\preprint{}

%Title of paper
\title{Polarity in Gapless Atomic Switch Networks}

\author{Forrest C. Sheldon}
\email[]{fsheldon@ucsd.edu}
%\homepage[]{Your web page}
%\thanks{}
%\altaffiliation{}


\author{Massimiliano Di Ventra}
\email[]{diventra@physics.ucsd.edu}
%\homepage[]{Your web page}
%\thanks{}
%\altaffiliation{}
\affiliation{Department of Physics, University of California San Diego,
La Jolla, California 92093, USA}

%Collaboration name if desired (requires use of superscriptaddress
%option in \documentclass). \noaffiliation is required (may also be
%used with the \author command).
%\collaboration can be followed by \email, \homepage, \thanks as well.
%\collaboration{}
%\noaffiliation

\date{\today}

% insert suggested PACS numbers in braces on next line
\pacs{}
% insert suggested keywords - APS authors don't need to do this
%\keywords{}

%\maketitle must follow title, authors, abstract, \pacs, and \keywords
\maketitle

Memristors are generally polar devices; a bias applied in one direction
leads to the conductive ON
state and a bias in the reverse direction back to the insulating OFF
state.  In constructing a model for
the atomic switch networks of \cite{Avizienis2012, Stieg2014, Stieg14}
we assumed that the polarity of the memristors in the network coincided
with the direction of the current flow from the boundaries. When the
network was biased from one direction, all devices transitioned from
OFF to ON and in the reverse direction from ON to OFF. In addition to
simplifying the model, we believe this to be a good approximation for
the behavior of these networks.

The $Ag/Ag_2 S/Ag$ atomic switches formed in these networks are gapless
type devices (see Hasegawa \textit{et al.} \cite{Hasegawa2012} for a
review of types of atomic switches).  Their symmetric metallic
configuration (typically gapless switches have two differing metallic
electrodes \textit{e.g.} $Ag/Ag_2 S/Pt$) suggests that at
the point of formation within the network, nothing has selected a
preferred direction within the switch.  Polarity is instead instilled
through a `formation' step in which a bias is applied to the network
causing filament structures to form in the switches throughout. After
a joule heating assisted dissolution of the thinnest part of the filament,
the junctions display bipolar resistive switching.  The polarity of the
internal switches is thus determined by the direction of current propagation
as we have assumed in the model.  That this must be true in the experimental
systems is evident from the fact that the network undergoes resistive
switching as a whole.  Without a majority of switch polarities coinciding
with the direction of currents from the boundaries, the network would switch
between identical states with half the switches in the OFF state and half ON
and thus not display the pinched hysteresis observed in experiment.

% figures should be put into the text as floats.
% Use the graphics or graphicx packages (distributed with LaTeX2e)
% and the \includegraphics macro defined in those packages.
% See the LaTeX Graphics Companion by Michel Goosens, Sebastian Rahtz,
% and Frank Mittelbach for instance.
%
% Here is an example of the general form of a figure:
% Fill in the caption in the braces of the \caption{} command. Put the label
% that you will use with \ref{} command in the braces of the \label{} command.
% Use the figure* environment if the figure should span across the
% entire page. There is no need to do explicit centering.

% \begin{figure}
% \includegraphics{}%
% \caption{\label{}}
% \end{figure}

% Surround figure environment with turnpage environment for landscape
% figure
% \begin{turnpage}
% \begin{figure}
% \includegraphics{}%
% \caption{\label{}}
% \end{figure}
% \end{turnpage}

% tables should appear as floats within the text
%
% Here is an example of the general form of a table:
% Fill in the caption in the braces of the \caption{} command. Put the label
% that you will use with \ref{} command in the braces of the \label{} command.
% Insert the column specifiers (l, r, c, d, etc.) in the empty braces of the
% \begin{tabular}{} command.
% The ruledtabular enviroment adds doubled rules to table and sets a
% reasonable default table settings.
% Use the table* environment to get a full-width table in two-column
% Add \usepackage{longtable} and the longtable (or longtable*}
% environment for nicely formatted long tables. Or use the the [H]
% placement option to break a long table (with less control than 
% in longtable).
% \begin{table}%[H] add [H] placement to break table across pages
% \caption{\label{}}
% \begin{ruledtabular}
% \begin{tabular}{}
% Lines of table here ending with \\
% \end{tabular}
% \end{ruledtabular}
% \end{table}

% Surround table environment with turnpage environment for landscape
% table
% \begin{turnpage}
% \begin{table}
% \caption{\label{}}
% \begin{ruledtabular}
% \begin{tabular}{}
% \end{tabular}
% \end{ruledtabular}
% \end{table}
% \end{turnpage}

% Specify following sections are appendices. Use \appendix* if there
% only one appendix.
%\appendix
%\section{}

% If you have acknowledgments, this puts in the proper section head.
%\begin{acknowledgments}
% put your acknowledgments here.
%\end{acknowledgments}

% Create the reference section using BibTeX:
\bibliography{/Users/forrestsheldon/Documents/BibTeX/Advancement_PT_Memnets}

\end{document}
%
% ****** End of file apstemplate.tex ******

